\input texbase

\titulo{RSS - Seminários do IME}
\materia{MAC0448 - Programação para Redes de Computadores}

\aluno{Diogo Haruki Kykuta}{6879613}
\aluno{Fernando Omar Aluani}{6797226}

\begin{document}
\cabecalho

\section{Manual do Usuário}
Ao abrir, aparecerá uma lista de notícias RSS, correspondentes a cada
palestra que será dada. Ao clicar, você é redirecionado para a notícia.

É exibida uma única lista, com todos os seminários dos departamentos
escolhidos (DCC, MAT, MAE e MAP). Assim, se todos estiverem ativos, a
inicialização do aplicativo será lenta.


\section{Estrutura do Código}
Criamos um módulo apenas para cuidar da persistência dos dados. Isso é feito
por meio de anotações na classe do modelo. Foi inspirado em JPA. Apesar de
existirem alguns frameworks por ai que disponibilizem isso, acreditamos que
seria um estudo interessante implementarmos nós mesmos.

Temos 3 Activities:
\begin{itemize}
    \item \textbf{MainActivity}\\
A entrada do aplicativo. Ela
apresenta um layout de lista, para apresentarmos a lista de palestras
que foi baixada.
    \item \textbf{ConfiguracoesActivity} \\
Tela de configurações, onde são escolhidos os canais (DCC, MAP, MAE, MAT)
desejados.
    \item \textbf{DetailsActivity}\\
Tela para apresentação da notícia. Nela, temos um botão para registrar a 
palestra no calendário, o corpo da notícia e um botão para abrir o navegador
padrão na URL vinda no link do feed.
\end{itemize}

\end{document}
