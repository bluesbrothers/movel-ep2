\input texbase

\titulo{Bandex}
\materia{MAC0448 - Programação para Redes de Computadores}

\aluno{Diogo Haruki Kykuta}{6879613}
\aluno{Fernando Omar Aluani}{6797226}

% AINDA VOU MUDAR O TEMPLATE. MAS AQUI EH O MANUAL DO USUARIO

\begin{document}
\cabecalho

\section{Manual do Usuário}
Ao abrir o aplicativo, o usuário é levado para a tela de escolha do bandejão.

\subsection{Tela de escolha do Bandejão}
Aqui, o usuário pode escolher entre os 4 bandejões: Central, Prefeitura, Física e Química.

Ao clicar em alguma das opções, o usuário é levado a Tela de Cardápio, já mostrando a próxima refeição. Essa próxima refeição é escolhida usando o dia atual e a hora atual (até as 8 da manhã: café da manhã, das 8 às 15: almoço, das 15 à meia noite: janta).

\subsection{Tela de Cardápio}
Aqui é apresentado o dia referente a refeição e qual refeição (café-da-manhã, almoço ou janta), o cardápio em si e 3 botões:
\begin{itemize}
\item \textbf{Cardápio da Semana} \\
Mostra a Tela de Cardápio da Semana do bandejão escolhido
\item \textbf{Localização no Mapa} \\
Mostra a Tela de Mapa, mostrando um marcador na localização do bandejão.
\item \textbf{Comentários} \\
Mostra a Tela de Comentários se já estiver logado. Caso contrário, vai para a Tela de Login.
\end{itemize}

\subsection{Tela de Cardápio da Semana}
Mostra o cardápio da semana, ordenado por: Dia e refeição (Segunda almoço, segunda janta, terça almoço, terça janta, ...)
Clicar em um dos itens da lista leva a Tela de Cardápio referente àquela refeição.

\subsection{Tela de Login}
Tela de login clássica, aceitando apenas números para o número usp. Ao logar, é direcionado para a Tela de Comentários.

\subsection{Tela de Comentários}
Apresenta a lista de comentários, onde cada item tem na primeira linha o login do commenter e nas demais o comentário.

No final da lista, tem um botão para postar comentários, que leva a Tela de Postagem.

No menu desta tela, existe a opção de deslogar.

\subsection{Tela de Postagem}
Nesta tela tem um campo de texto para digitar seu comentário e um botão para enviar.

\section{Estrutura do Código}
Apresentação de forma superficial das classes principais do código.
\subsection{Activities}
\subsubsection{ComentariosActivity}
Activity para exibir os comentários sobre a refeição. É um ListActivity com um ListAdapter customizado (ComentariosAdapter). Depende do id da refeição no Intent.

\subsubsection{DetailsActivity}
Activity para apresentar o cardápio do dia. Depende do dia e da refeição a que estamos nos referindo e do id do bandejão no Intent.

\subsubsection{FullCardapioActivity}
Activity para apresentar todo o cardápio da semana. É um ListActivity com um ListAdapter customizado (RefeicoesAdapter).
Depende do id do bandejão no Intent.

\subsubsection{LoginActivity}
Activity para fazer login. Boa parte dela foi gerada automaticamente pelo ADT ;)

\subsubsection{MainActivity}
Activity principal, aqui temos basicamente os 4 botões para escolher o bandejão.

\subsubsection{MapActivity}
???

\subsubsection{PostarActivity}
Activity para postar comentários. Depende do id da refeição no Intent.

\subsection{Adapters}
\subsubsection{ComentariosAdapter}
Um Adapter para utilizar uma view customizada, para apresentar o nick do commenter em uma linha acima, em negrito, e o comentário embaixo, com fonte normal.
\subsubsection{RefeicoesAdapter}
Um Adapter para utilizar uma view customizada, para apresentar a data e a refeição referente a um item do cardápio da semana em uma lista acima.

\subsection{Modelos}
\subsubsection{CardapioDia}
Modelo para guardar uma refeição de um dia.

\subsubsection{UltimoCardapio}
Modelo para guardar o último cardápio baixado de um bandejão. Utilizamos para isso a semana do ano que corresponde aquele cardápio baixado. Uma observação importante é que consideramos que a SEGUNDA-FEIRA é o primeiro dia da semana.

\subsection{Managers}
\subsubsection{ComentariosManager}
Gerenciador de Comentários. Tem métodos para fazer o parse dos comentários e para postar um novo comentário.

\subsubsection{ContentManager}
Gerenciador de Cardápios. Tem métodos para fazer o parse dos cardápios e salvar os novos cardápios.

\subsubsection{StoaManager}
Gerenciador da conta no STOA. Tem métodos para fazer o login, verificar se está logado e deslogar.

\subsection{Utilidades}
Tem alguns Enums, uma classe de constantes e algumas classes auxiliares que listarei abaixo.
\subsubsection{BandexCalculator}
Classe que possui métodos estáticos para calcular a próxima refeição (levando em consideração dia e hora atuais), a semana referente a uma data e gerar a String que servirá como título da Tela de Cardápio.

\subsubsection{BandexComment}
Uma classe para representar um comentário. Possui duas Strings: o commenter e a message.

\subsubsection{CardapioSemana}
Uma classe para representar o cardápio de uma semana. Serve para não precisar se preocupar com o Map dentro de outro Map que guarda os cardápios, provendo métodos para um uso mais intuitivo no código.

\subsubsection{ComentarioAsyn}
Classe assíncrona para postar um comentário sem travar a aplicação.

\subsubsection{MarkerInfo}
???



\end{document}