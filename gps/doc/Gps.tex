\input texbase

\titulo{Usando GPS}
\materia{MAC0448 - Programação para Redes de Computadores}

\aluno{Diogo Haruki Kykuta}{6879613}
\aluno{Fernando Omar Aluani}{6797226}

\begin{document}
\cabecalho

\section{Manual do Usuário}
Ao abrir o aplicativo, o usuário irá ver a tela principal. 

\subsection{Tela Principal}
A tela principal mostra de cara qual circular está selecionado (o padrão inicial é o Circular 1 (8012)),
e um mapa centralizado na USP.

O mapa sempre mostra a rota do circular selecionado (em vermelho para o 8012, verde para o 8022) e a posição
atual do usuário (um ponto ou seta azul).

Ao receber a posição do usuário, ele irá calcular qual o ponto na rota mais próximo ao usuário, e o próximo horário
que esse circular está previsto para passar nesse ponto. Ao fazer isso, o seguinte acontece:
\begin{itemize}
    \item O texto acima do mapa que fala qual circular está selecionado
        irá também mostrar o horário de chegada prevista do próximo ônibus no ponto da rota mais próximo ao usuário;
    \item O mapa irá mostrar um marcador no ponto mais próximo. Ao clicar em tal marcador ele irá mostrar
        uma janelinha (como um balão) em cima do marcador, mostrando os dados que foram calculados: distância
        do terminal até esse ponto, tempo de chegada, horário de saída do terminal do ônibus, horário previsto de chegada
        do ônibus nesse ponto e horário previsto que esse ônibus vai chegar de volta ao terminal (após dar a volta).
\end{itemize}

É possível também fazer um clique-longo (toque a tela e segure o dedo por uns segundos) no mapa. Ao fazer isso,
será criado um marcador no ponto da rota mais próximo ao ponto do mapa onde você clicou. Esse marcador irá mostrar as
mesmas informações que o marcador mais próximo do usuário, porém relativas a esse marcador especifico.
Podem ser colocados quantos marcadores o usuário quiser.

A cada intervalo de 1 minuto que o app esteja rodando nesta tela, ele irá atualizar os dados dos ônibus (horários e etc)
de cada marcador no mapa.
Para remover os marcadores do usuário, basta selecionar um circular no menu de opções (explicado abaixo). Não é possível
remover o marcador do ponto mais próximo ao usuário, porém ele irá sumir ao selecionar um circular, e só reaparecerá
se foi possível detectar a sua posição.

O aplicativo sempre tentará pegar uma posição precisa do usuário, mas isso depende das capacidades e configurações
do dispositivo. Os métodos existentes são:
\begin{itemize}
    \item \textbf{Sinal de Celular}: pega a posição pela triangularização do sinal. É pouco precisa, mas é rápida,
        consome pouca energia e está quase sempre disponível;
    \item \textbf{Dados (3G/WiFi)}: usa a internet para cálculo da posição. Normalmente é precisa, mas velocidade/consumo
        de energia depende da origem dos dados, e também pode haver outros gastos (com rede de dados 3G por exemplo).
    \item \textbf{GPS}: se o GPS estiver ligado, usa os satélites GPS para triangularização. É o método mais preciso,
        mas gasta mais energia, pode demorar um pouco inicialmente para conectar-se aos satélites, e o ambiente
        pode influenciar adversivamente a conexão.
\end{itemize}

O mapa (Google Maps) necessita de conexão à internet para baixar os dados do mapa. No entanto, ele faz cache 
do mapa. Logo ele só irá necessitar internet de fato na primeira vez que o app é aberto, ou de tempos em tempos
depois se o cache for limpo. Como o mapa é aberto já centralizado na USP, é certeza que ele ira cachear as
regiões importantes para o aplicativo, e uma vez que eles tenham sido carregados, dá pra ver o mapa da USP
sem conexão à internet.
O mapa também aceita gestos intuitivos com os dedos para movimentação do mapa, zoom, etc.

O menu de opções tem 4 opções:
\begin{itemize}
    \item \textbf{Circular 1 (8012)}: muda o circular selecionado para este;
    \item \textbf{Circular 2 (8022)}: muda o circular selecionado para este;
    \item \textbf{Centralizar na USP}: move o mapa e altera o zoom para centralizar a camêra na USP;
    \item \textbf{Tabela de Horários}: abre a tela da tabela de horários.
\end{itemize}

\subsection{Tela da Tabela de Horários}
Essa tela mostra:
\begin{itemize}
    \item o circular e dia da semana selecionado;
    \item o horário que o próximo ônibus dessa linha irá sair do terminal;
    \item a duração (em minutos) da viagem de um ônibus (terminal -> volta na USP -> terminal) nesse dia;
    \item a lista de horários de saída do terminal dos ônibus nesse dia, ordenados do mais cedo ao mais tarde;
\end{itemize}

O menu de opções possibilita a escolha do circular e a escolha do dia da semana.

É possível escolher para mostrar ambos circulares. Nesse caso, o horário do próximo a sair será o mais próximo
entre as duas linhas; a duração irá mostrar, separadamente, a duração das duas linhas; e a lista de horários
irá mostrar os horários das duas linhas juntas propriamente ordenado, sendo que cada item na lista irá mostrar o
código da linha além do horário, e o fundo do item será colorido assim como a rota dessa linha no mapa.

A cada intervalo de 1 minuto que o app esteja rodando nesta tela, ele irá atualizar o horário do próximo ônibus
que irá sair.

Para voltar para a tela principal basta apertar o botão \textit{Back} do seu dispisitivo.

\section{Estrutura do Código}
O código escrito (não incluindo o gerado pelo ADT) foi dividido
em 3 partes, relacionadas aos pacotes. Fora isso, também temos os
recursos da aplicação.

\subsection{Pacote principal (fefzjon.ep2.gps)}
Aqui estão as classes principais do aplicativo:
\begin{itemize}
    \item \textbf{MainActivity}: é a \textit{activity} principal do aplicativo. A primeira a ser executada. Mostra o mapa com a rota do circular selecionado.
        Ao receber a localização do usuário, ele mostra no mapa o ponto na rota mais próximo ao usuário, e calcula os horários relacionados.
        O menu de opções permite selecionar qual circular é mostrado, centralizar o mapa na USP e mostrar as tabelas completas de horários de saída dos
        circulares no terminal Butantã.
    \item \textbf{TimeTableActivity}: \textit{activity} secundária que mostra a duração da viagem e horários de saida do terminal para um dado circular
        em um dado dia. O menu de opções permite escolher o circular e o dia.
    \item \textbf{MarkerInfo}: classe que implementa uma interface do GoogleMaps para gerar o View mostrado ao usuário quando ele clica num marcador no mapa.
    \item \textbf{MarkerInfo.MarkData}: estrutura para guardar os dados de um ônibus que são mostrados por um marcador no mapa.
\end{itemize}

\subsection{Pacote de Utilidade (fefzjon.ep2.gps.utilities)}
Aqui estão as classes de utilidades usadas pelo aplicativo. Em sua maior parte são classes \textit{singleton} que
gerenciam os dados relacionados aos circulares (rotas e horários).
\begin{itemize}
    \item \textbf{ConnectionHandler}: classe auxiliar que gerencia criação, uso e callbacks da API de Location do Android, a qual (entre outras coisas)
        envia à sua aplicação a posição do usuário. Ela sempre tenta adquirir uma posição precisa em intervalos curtos de tempo (5 segundos), mas
        isso pode mudar de acordo com os recursos do dispositivo.
    \item \textbf{RouteManager}: classe que trata o acesso e outras operações comuns para as rotas (sequência de pontos de latitude/longitude) dos circulares.
    \item \textbf{TimetableManager}: classe que trata o acesso a e outras operações comuns para os horários (duração e horários de saída) dos circulares.
    \item \textbf{Constants}: classe que define alguns valores constantes usados pelo app.
    \item \textbf{Timer}: classe que possibilita a chamada de um callback repetidas vezes, com um intervalo de tempo entre cada chamada.
    \item \textbf{TimerCallback}: interface que define o método de callback que o Timer executa.
    \item \textbf{TimeTableAdapter}: classe que extende a classe básica de adaptador de dados para lista. Ela provê para a ListView do TimeTableActivity
          os items da tabela de horários dos circulares.
\end{itemize}

\subsection{Pacote de Diálogos (fefzjon.ep2.gps.dialog)}
Aqui estão as classes de dialogos usadas pelo projeto. Todas extendem a \textit{DialogFragment} e implementam
um dialogo simples para escolha de 1 item dentre uma lista de possibilidades.
\begin{itemize}
    \item \textbf{SelectBUSPDialogFragment}: dialogo para seleção do circular, de acordo com o código do ônibus.
    \item \textbf{SelectBUSPDialogFragment.BUSPDialogListener}: interface que define o método de callback chamado 
          pelo SelectBUSPDialogFragment quando uma escolha é feita.
    \item \textbf{SelectDayTypeDialogFragment}: dialogo para seleção do dia da semana (dia útil, sábado ou domingo).
    \item \textbf{SelectDayTypeDialogFragment.DayTypeDialogListener}: interface que define o método de callback chamado
          pelo SelectDayTypeDialogFragment quando uma escolha é feita.
\end{itemize}

\subsection{Recursos}
Fora os recursos padrão do Android, como o \textit{strings}, \textit{layout}'s e \textit{menu}'s,
só temos um recurso notável a mais: o \textbf{circulares.xml}, em \textit{res/values}.

Esse recurso contém todos os dados dos circulares: latitudes e longitudes dos pontos da rota,
duração da viagem e horários de saída para cada circular em cada dia da semana.

Todos esses dados foram pegos da tabela disponibilizada pelo monitor, que os adquiriu do site
da SPTrans. A unica diferença é que pontos iguais seguidos que existiam na lista de pontos
original foram reduzidos a um único ponto.


\end{document}
